% \iffalse meta-comment
% 
% Copyright (C) 2022 by SiyuWu
% ----------------------------
%
% This file may be distributed and/or modified under the
% conditions of the LaTeX Project Public License, either version 1.3 
% of this license or (at your option) any later version.
% The latest version of this license is in:
%
% http://www.latex-project.org/lppl.txt
%
% and version 1.3 or later is part of all distributions of LaTeX
% version 2005/12/01 or later.
% 
% \fi
% 
% \iffalse
%<*driver>
\ProvidesFile{mathexercise.dtx}
%</driver>
%<class>\NeedsTeXFormat{LaTeX2e}[2022/12/10]
%<class>\ProvidesClass{mathexercise}
%<*class>
  [2023/02/22 v1.1 class file to write mathematical exercise solutions]
%</class>

%<*driver>
\documentclass{ltxdoc}
\usepackage[fontset=fandol]{ctex}
\usepackage{enumitem}
\usepackage{mathtools}
\usepackage{booktabs}
\usepackage[toc]{multitoc}
\usepackage[left=4cm]{geometry}
\setlist{nosep}
\DeclarePairedDelimiterX{\lrangle}[1]{\langle}{\rangle}{#1}
\DeclarePairedDelimiterX{\innerp}[2]{\langle}{\rangle}{#1,#2}
\DeclarePairedDelimiterX{\norm}[1]{\lVert}{\rVert}{#1}
\newcommand{\cls}[1]{\textsf{#1}}
\newcommand{\option}[1]{\texttt{#1}}
\newcommand{\env}[1]{\texttt{#1}}
\newcommand{\pkg}[1]{\textsf{#1}}
\newcommand{\oargBracket}[1]{{\tt[}\meta{#1}{\tt]}}
\newcommand{\oargAngle}[1]{{\tt<}\meta{#1}{\tt>}}
\EnableCrossrefs
\CodelineIndex
\RecordChanges
\begin{document}
  \DocInput{\jobname.dtx}
\end{document}
%</driver>
% \fi
%
% \CheckSum{348}
% 
% \CharacterTable
%  {Upper-case    \A\B\C\D\E\F\G\H\I\J\K\L\M\N\O\P\Q\R\S\T\U\V\W\X\Y\Z
%   Lower-case    \a\b\c\d\e\f\g\h\i\j\k\l\m\n\o\p\q\r\s\t\u\v\w\x\y\z
%   Digits        \0\1\2\3\4\5\6\7\8\9
%   Exclamation   \!     Double quote  \"     Hash (number) \#
%   Dollar        \$     Percent       \%     Ampersand     \&
%   Acute accent  \'     Left paren    \(     Right paren   \)
%   Asterisk      \*     Plus          \+     Comma         \,
%   Minus         \-     Point         \.     Solidus       \/
%   Colon         \:     Semicolon     \;     Less than     \<
%   Equals        \=     Greater than  \>     Question mark \?
%   Commercial at \@     Left bracket  \[     Backslash     \\
%   Right bracket \]     Circumflex    \^     Underscore    \_
%   Grave accent  \`     Left brace    \{     Vertical bar  \|
%   Right brace   \}     Tilde         \~}
% 
% \changes{v1.0}{2022/12/12}{initial version}
% \DoNotIndex{\arabic,\begin,\bfseries,\bigskip,\ccwd,\centering}
% \DoNotIndex{\chinese,\clearpage,\counterwithin,\ctexset,\CurrentOption}
% \DoNotIndex{\DeclareMathOperator,\DeclareOption}
% \DoNotIndex{\def,\definecolor,\dfrac,\draw}
% \DoNotIndex{\else,\end,\endtrivlist,\ExecuteOptions}
% \DoNotIndex{\fi,\fill,\gdef,\hfill,\href,\hspace,\huge,\hypersetup}
% \DoNotIndex{\item,\itshape,\labelsep,\linespread,\linewidth}
% \DoNotIndex{\LoadClass,\maketitle,\mathop,\mathrm,\medskip}
% \DoNotIndex{\newcommand,\newcounter}
% \GetFileInfo{mathexercise.dtx}
% 
% \title{\bfseries\cls{mathexercise} 文档类使用手册\thanks{This document
%   corresponds to \cls{mathexercise}~\fileversion, dated \filedate.}}
% \author{Siyu Wu \\ \texttt{sywumath@gmail.com}}
% 
% \maketitle
%
% \tableofcontents
% 
% \section{介绍}
% 
% \cls{mathexercise} 文档类主要用于编写数学习题解.
% 
% 
% \section{文档类选项}
% 
% \cls{mathexercise} 文档类有下列选项:
% \begin{description}
%   \item [\option{withinchap}] (default) 此选项使得 \env{exercise} 环境的计数器随 chapter 自增,
%     当开始新的 chapter, 计数器的值重置为零.
%   \item [\option{withinsec}] 此选项使得 \env{exercise} 环境的计数器随 section 自增,
%     当开始新的 section, 计数器的值重置为零.
%   \item [\option{chinese}] 文档类默认不提供中文支持, 使用 \option{chinese} 选项后将调用 \pkg{ctex}
%     宏包并进行中文适配. 默认情况下建议使用 \texttt{pdflatex} 编译命令, 使用 \option{chinese}
%     选项后请使用 \texttt{xelatex} 或者 \texttt{lualatex} 编译命令.
%   \item [\option{tikzcover}] 文档类默认情况下不使用 tikz 定制封面, 
%     如果需要此功能, 请使用 \option{tikzcover} 选项.
% \end{description}
% 
% 
% \section{导言区声明}
% 
% 文档类需要用户在导言区输入自定义信息, 如果使用了 \option{tikzcover} 选项,
% 需要输入类似于下列的信息:
% \begin{verbatim}
%     \title{PDE Solutions}
%     \author{wsy}
%     \textbook{Partial Differential Equations}
%     \textbookauthor{Evans}
%     \textbookversion{Second Edition}
% \end{verbatim}
% 如果未使用 \option{tikzcover} 选项, 则只需要和标准文档类一样只输入
% title 和 author 信息即可.
% 随后, 使用 \verb|\maketitle| 命令便可以生成封面.
% 
% 
% \section{环境}
% 
% \DescribeEnv{exercise}
% \DescribeEnv{proof}
% \DescribeEnv{solution}
% \DescribeEnv{theorem}
% 文档类主要提供了下列环境:
% \begin{description}
%   \item [\env{exercise}] 此环境接受两个可选参数, 分别为 \oarg{integer}
%     和 \oargAngle{text}. 可选参数 \oarg{integer} 将当前题目的题号设置为 \meta{integer},
%     且后续题目的题号默认从 $\meta{integer}+1$ 开始增加. 使用可选参数 \oargAngle{text} 将会在题号后面
%     以加粗字体显示 \meta{text} 并以圆括号将其包裹.
%   \item [\env{proof}] 证明环境
%   \item [\env{solution}] 解答环境
%   \item [\env{theorem}] 定理环境
%   \item [\env{lemma}] 引理环境
%   \item [\env{example}] 例环境
%   \item [\env{remark}] 注环境
%   \item [\env{corollary}] 推论环境
%   \item [\env{proposition}] 命题环境
%   \item [\env{note}] 笔记环境
%   \item [\env{tasks}] 由 \pkg{tasks} 宏包提供, 具体使用方法参考其手册
% \end{description}
% 
% 注意: \env{proof}、\env{solution}、\env{theorem}、\env{lemma}、\env{example}、\env{remark}、
% \env{corollary}、\env{note} 环境的 Heading 受 \option{chinese} 选项影响.
% 
% 
% \section{命令}
% 
% 文档类定义了如下命令, 方便用户调用:
% 
% \subsection{无参数命令}
% 
% \begin{center}
%   \begin{tabular}{ccc}
%     \toprule
%       Command & Effect & Meaning \\
%     \midrule
%       \verb|\e|  & $\mathrm{e}$ & Euler's number \\
%       \verb|\T|  & $\mathrm{T}$ & use for the transpose of matrix \\
%       \verb|\upi| & $\mathrm{i}$ & imaginary unit \\
%       \verb|\id|  & $\mathrm{id}$ & identity map \\
%       \verb|\diff| & $\mathrm{d}$ & differential operator \\
%       \verb|\codim| & $\operatorname{codim}$ & codimension \\
%       \verb|\conv| & $\operatorname{conv}$ & convex hall \\
%       \verb|\diag| & $\operatorname{diag}$ & diagonal \\
%       \verb|\diam| & $\operatorname{diam}$ & diameter \\
%       \verb|\dist| & $\operatorname{dist}$ & distance \\
%       \verb|\Int|  & $\operatorname{Int}$  & interior of set \\
%       \verb|\rank| & $\operatorname{rank}$ & rank \\
%       \verb|\sgn|  & $\operatorname{sgn}$ & sign \\
%       \verb|\supp| & $\operatorname{supp}$ & support \\
%       \verb|\lcm| & $\operatorname{lcm}$ & least common multiple \\
%       \verb|\Span| & $\operatorname{span}$ & span \\
%       \verb|\Re| (redefine) & $\operatorname{Re}$ & real part \\
%       \verb|\Im| (redefine) & $\operatorname{Im}$ & imaginary part \\
%       \verb|\weakconverge|  & $\rightharpoonup$   & weakly converge to \\ 
%     \bottomrule
%   \end{tabular}
% \end{center}
% 
% 
% \subsection{有参数命令}
% 
% \begin{center}
%   \begin{tabular}{cccc}
%     \toprule
%       Command & Example & Effect & Meaning \\
%     \midrule
%       \verb|\closure| & \verb|\closure{A}| & $\overline{A}$ & closure of set \\
%       \verb|\conjugate| & \verb|\conjugate{z}| & $\overline{z}$ & conjugate number \\
%       \verb|\lrangle| & \verb|\lrangle{g}| & $\lrangle{g}$ &  \\
%       \verb|\innerp| & \verb|\innerp{a}{b}| & $\innerp{a}{b}$ & inner product \\
%       \verb|\norm| & \verb|\norm{v}| & $\norm{v}$ & norm \\
%     \bottomrule
%   \end{tabular}
% \end{center}
% 
% 
% \section{宏包依赖}
% 
% \cls{mathexercise} 文档类预加载了下列宏包用于提供各种常用功能:
% \begin{itemize}
%   \item \pkg{ctex} (取决于 \option{chinese} 选项)
%   \item \pkg{indentfirst}
%   \item \pkg{enumitem}
%   \item \pkg{mathtools} (\pkg{mathtools} 宏包预加载 \pkg{amsmath} 宏包, 
%     本文档类依赖 \pkg{mathtools} 创建命令 \verb|\lrangle|、\verb|\innerp|、\verb|\norm|)
%   \item \pkg{amssymb}
%   \item \pkg{amsthm}
%   \item \pkg{extarrows}
%   \item \pkg{xcolor}
%   \item \pkg{tcolorbox}
%   \item \pkg{geometry}
%   \item \pkg{tikz}
%   \item \pkg{tasks}
%   \item \pkg{hyperref}
%   \item \pkg{fixdif} (提供微分算符 \verb|\d|)
% \end{itemize}
% \changes{v1.1}{2023/02/22}{调用 \pkg{fixdif} 包}
% 如果需要使用其它宏包, 直接在主文件导言区通过 \verb|\usepackage{}| 命令添加即可.
% 
% \StopEventually{\PrintChanges\PrintIndex}
%
% \section{Source Code}
% 
% \begin{macro}{\if@chinesesupport}
% \begin{macro}{\if@tikzcover}
% 是否提供中文支持以及是否使用 TikZ 绘制封面。
%    \begin{macrocode}
\newif\if@chinesesupport
\newif\if@tikzcover
\@chinesesupportfalse
\@tikzcoverfalse
%    \end{macrocode}
% \end{macro}
% \end{macro}
% 
% \begin{macro}{withinchap}
% \begin{macro}{withinsec}
% \begin{macro}{tikzcover}
% \begin{macro}{chinese}
% 声明选项,处理选项。
%    \begin{macrocode}
\DeclareOption{withinchap}{\def\exercounter@within{chapter}}
\DeclareOption{withinsec}{\def\exercounter@within{section}}
\DeclareOption{tikzcover}{\@tikzcovertrue}
\DeclareOption{chinese}{\@chinesesupporttrue}
\DeclareOption*{\PassOptionsToClass{\CurrentOption}{report}}
\ExecuteOptions{withinchap}
\ProcessOptions\relax
%    \end{macrocode}
% \end{macro}
% \end{macro}
% \end{macro}
% \end{macro}
% 加载 \cls{report} 文档类
%    \begin{macrocode}
\LoadClass{report}
%    \end{macrocode}
% 加载所需宏包
%    \begin{macrocode}
\RequirePackage{indentfirst}
\RequirePackage[shortlabels]{enumitem}
\RequirePackage{mathtools}
\RequirePackage{amssymb}
\RequirePackage{amsthm}
\RequirePackage{fixdif}
\RequirePackage{extarrows}
\RequirePackage{xcolor}
\RequirePackage{tcolorbox}
\RequirePackage[width=14cm]{geometry}
\RequirePackage{tikz}
\usetikzlibrary{intersections,decorations.text}
%    \end{macrocode}
% 定义颜色值。
%    \begin{macrocode}
\definecolor{c1}{RGB}{62,  97,  127}
\definecolor{c2}{RGB}{104, 182, 182}
\definecolor{c3}{RGB}{107, 190, 190}
\definecolor{c4}{RGB}{100, 172, 174}
\definecolor{c5}{RGB}{95,  162, 162}
%    \end{macrocode}
% 根据 \pkg{amsthm} 宏包,新建定理风格。
%    \begin{macrocode}
\newtheoremstyle{mytheorem}{3pt}{3pt}{\rmfamily}{\parindent}{\bfseries}{}%
  {.5em}{\thmname{#1}\thmnumber{ #2}\thmnote{ (#3)}}
%    \end{macrocode}
% 根据是否提供中文支持分别进行中西文设定。
%    \begin{macrocode}
\if@chinesesupport
  \RequirePackage[heading=true, fontset = fandol]{ctex}
  \ctexset {
    chapter = {
      beforeskip = 0pt,
      fixskip = true,
      format = \Huge\bfseries,
      nameformat = \rule{\linewidth}{1bp}\par\bigskip\hfill\chapternamebox,
      number = \arabic{chapter},
      aftername = \par\medskip,
      aftertitle = \par\bigskip\nointerlineskip\rule{\linewidth}{2bp}\par}
  }
  \newcommand\chapternamebox[1]{%
  \parbox{\ccwd}{\linespread{1}\selectfont\centering #1}}
  \newcommand{\solutionname}{解}
  %theorem environment
  \theoremstyle{mytheorem}\newtheorem{theorem}{定理}
  \theoremstyle{mytheorem}\newtheorem{lemma}{引理}
  \theoremstyle{mytheorem}\newtheorem{example}{例}
  \theoremstyle{mytheorem}\newtheorem*{remark}{注}
  \theoremstyle{mytheorem}\newtheorem*{corollary}{推论}
  \theoremstyle{mytheorem}\newtheorem{proposition}{命题}
  \theoremstyle{mytheorem}\newtheorem{note}{笔记}
  \newcommand{\authorname}{作者}
  \newcommand{\textbookname}{教材}
  \newcommand{\proofnamestyle}{\bfseries}
  \newcommand{\proofpunct}{:}
\else
  \newcommand{\solutionname}{Solution}
  \theoremstyle{mytheorem}\newtheorem{theorem}{Theorem}
  \theoremstyle{mytheorem}\newtheorem{lemma}{Lemma}
  \theoremstyle{mytheorem}\newtheorem{example}{Example}
  \theoremstyle{mytheorem}\newtheorem*{remark}{Remark}
  \theoremstyle{mytheorem}\newtheorem*{corollary}{Corollary}
  \theoremstyle{mytheorem}\newtheorem{proposition}{Proposition}
  \theoremstyle{definition}\newtheorem{note}{Note}[chapter]
  \newcommand{\authorname}{Author}
  \newcommand{\textbookname}{Textbook}
  \newcommand{\proofnamestyle}{\itshape}
  \newcommand{\proofpunct}{.}
\fi
%    \end{macrocode}
% 
% \begin{macro}{\e}
% \begin{macro}{\T}
% \begin{macro}{\upi}
% \begin{macro}{\id}
% 罗马字体
%    \begin{macrocode}
\newcommand{\e}{\mathrm{e}}
\newcommand{\T}{\mathrm{T}}
\newcommand{\upi}{\mathrm{i}}
\newcommand{\id}{\mathrm{id}}
%    \end{macrocode}
% \end{macro}
% \end{macro}
% \end{macro}
% \end{macro}
%
% \begin{macro}{\weakconverge}
% \begin{macro}{\closure}
% \begin{macro}{\conjugate}
% 弱收敛、闭包、共轭。
%    \begin{macrocode}
\newcommand{\weakconverge}{\rightharpoonup}
\newcommand{\closure}[1]{\overline{#1}}
\newcommand{\conjugate}[1]{\overline{#1}}
%    \end{macrocode}
% \end{macro}
% \end{macro}
% \end{macro}
%
% \begin{macro}{\diff}
% 微分算符。
%    \begin{macrocode}
\newcommand{\diff}{\mathop{}\!\mathrm{d}}
%    \end{macrocode}
% \end{macro}
% 使用 |\DeclareMathOperator| 定义算符。
%    \begin{macrocode}
\DeclareMathOperator{\codim}{codim}
\DeclareMathOperator{\conv}{conv}
\DeclareMathOperator{\diag}{diag}
\DeclareMathOperator{\diam}{diam}
\DeclareMathOperator{\dist}{dist}
\DeclareMathOperator{\Int}{Int}
\DeclareMathOperator{\lcm}{lcm}
\DeclareMathOperator{\rank}{rank}
\DeclareMathOperator{\sgn}{sgn}
\DeclareMathOperator{\sign}{sign}
\DeclareMathOperator{\spt}{spt}
\DeclareMathOperator{\supp}{supp}
\DeclareMathOperator{\Span}{span}
\DeclareMathOperator{\tr}{tr}
%    \end{macrocode}
%
% \begin{macro}{\Rm}
% \begin{macro}{\Im}
% 重定义 |\Rm| 和 |\Im| 使其用于复数的实部和虚部。
%    \begin{macrocode}
\renewcommand{\Re}{\operatorname{Re}}
\renewcommand{\Im}{\operatorname{Im}}
%    \end{macrocode}
% \end{macro}
% \end{macro}
%
% \begin{macro}{\innerp}
% \begin{macro}{\lrangle}
% \begin{macro}{\norm}
% 利用 \pkg{mathtools} 宏包构建宏。
%    \begin{macrocode}
\DeclarePairedDelimiterX{\lrangle}[1]{\langle}{\rangle}{#1}
\DeclarePairedDelimiterX{\innerp}[2]{\langle}{\rangle}{#1,#2}
\DeclarePairedDelimiterX{\norm}[1]{\lVert}{\rVert}{#1}
%    \end{macrocode}
% \end{macro}
% \end{macro}
% \end{macro}
%
% \begin{macro}{\textbook}
% \begin{macro}{\textbookauthor}
% \begin{macro}{\textbookversion}
% 三个用户宏分别用于输入教材名、教材作者以及教材版本。
%    \begin{macrocode}
\newcommand{\textbook}[1]{\gdef\@textbook{#1}}
\newcommand{\textbookauthor}[1]{\gdef\@textbookauthor{#1}}
\newcommand{\textbookversion}[1]{\gdef\@textbookversion{#1}}
%    \end{macrocode}
% \end{macro}
% \end{macro}
% \end{macro}
%
% \begin{environment}{proof}
% \begin{environment}{solution}
% 重定义 |proof| 环境,新定义 |solution| 环境。
%    \begin{macrocode}
\renewenvironment{proof}[1][\proofname]{\par
  \pushQED{\qed}
  \normalfont\topsep1\p@\@plus6\p@\relax
  \trivlist\item\relax
	{\hspace*{\parindent}{\proofnamestyle #1}\@addpunct{\proofpunct}}
  \hspace\labelsep\ignorespaces
}{%
	\popQED\endtrivlist\@endpefalse
}

\newenvironment{solution}[1][\solutionname]{\par
	\pushQED{\qed}%
  \normalfont\topsep1\p@\@plus6\p@\relax
  \trivlist\item\relax
	{\hspace*{\parindent}{\proofnamestyle #1}\@addpunct{\proofpunct}}%
  \hspace\labelsep\ignorespaces
}{%
	\popQED\endtrivlist\@endpefalse
}
%    \end{macrocode}
% \end{environment}
% \end{environment}
%
% \begin{environment}{exercise}
% 定义 |exercise| 环境。
%    \begin{macrocode}
\newcounter{exercounter}
\counterwithin*{exercounter}{\exercounter@within}
\NewDocumentEnvironment{exercise}{o d<> +b}
  {%
    \IfNoValueTF{#1}
      {\stepcounter{exercounter}}
      {\setcounter{exercounter}{#1}}
    \par\textbf{\theexercounter.}\hspace{.333em}%
	\IfNoValueTF{#2}
    {#3}
    {\textbf{(#2)}\hspace{.333em}#3}
  }
  {\ignorespacesafterend}
%    \end{macrocode}
% \end{environment}
% \pkg{enumitem} 设定。
%    \begin{macrocode}
\setlist{nosep,left=\parindent}
%    \end{macrocode}
% \pkg{tasks} 宏包设置。
%    \begin{macrocode}
\RequirePackage{tasks}
\settasks{after-item-skip=0.5ex plus 0.5ex minus 1ex}
%    \end{macrocode}
% \pkg{hyperref} 宏包加载及设置。
%    \begin{macrocode}
\RequirePackage{hyperref}
\hypersetup{
  colorlinks,%
  linkcolor=red
}
%    \end{macrocode}
% \begin{macro}{\maketitle}
% 封面设置。
%    \begin{macrocode}
\if@tikzcover
  \renewcommand{\maketitle}{%
    \thispagestyle{empty}
    \begin{tikzpicture}[overlay,remember picture,font=\sffamily\bfseries]
      \draw[ultra thick,c4,name path=big arc] ([xshift=-2mm]current page.north) 
      arc(150:285:11)
      coordinate[pos=0.225] (x0);
      \begin{scope}
      \clip ([xshift=-2mm]current page.north) arc(150:285:11)
        --(current page.northeast);
      \fill[c4!50,opacity=0.25] ([xshift=4.55cm]x0) circle (4.55);
      \fill[c4!50,opacity=0.25] ([xshift=3.4cm]x0) circle (3.4);
      \fill[c4!50,opacity=0.25] ([xshift=2.25cm]x0) circle (2.25);
      \draw[ultra thick,c4!50] (x0) arc(-90:30:6.5);
      \draw[ultra thick,c4] (x0) arc(90:-30:8.75);
      \draw[ultra thick,c4!50,name path=arc1] (x0) arc(90:-90:4.675);
      \draw[ultra thick,c4!50] (x0) arc(90:-90:2.875);
      \path[name intersections={of=big arc and arc1,by=x1}];
      \draw[ultra thick,c4,name path=arc2] (x1) arc(135:-20:4.75);
      \draw[ultra thick,c4!50] (x1) arc(135:-20:8.75);
      \path[name intersections={of=big arc and arc2,by={aux,x2}}];
      \draw[ultra thick,c4!50] (x2) arc(180:50:2.25);
      \end{scope}
      \path[decoration={text along path,text color=c4,
              raise = -2.8ex,
              text  along path,
              text = {|\sffamily\bfseries|\@date},
              text align = center,
            },
            decorate
          ] ([xshift=-2mm]current page.north) arc(150:245:11);
      %
      \begin{scope}
      \path[clip,postaction={fill=c3}]
      ([xshift=2cm,yshift=-8cm]current page.center) rectangle ++ (4.2,7.7);
      \draw[c5,ultra thick,fill=c2] ([xshift=0.5cm,yshift=-8cm]current page.center)
        ([xshift=0.5cm,yshift=-8cm]current page.center)  arc(180:60:2)
        |- ++ (-3,6) --cycle;
      \draw[ultra thick,c5] ([xshift=-1.5cm,yshift=-8cm]current page.center)
      arc(180:0:2);
      \draw[ultra thick,c5] ([xshift=0.5cm,yshift=-8cm]current page.center)
      arc(180:0:2);
      \draw[ultra thick,c5] ([xshift=2.5cm,yshift=-8cm]current page.center)
      arc(180:0:2);
      \draw[ultra thick,c5] ([xshift=4.5cm,yshift=-8cm]current page.center)
      arc(180:0:2);
      \fill[red] ([xshift=2.5cm,yshift=-8cm]current page.center) +(60:2)
        circle(1.5mm) node[above right=2mm,text=c5!80!black]
        {$\rho:=\dfrac{1+\sqrt{-3}}{2}$};
      \end{scope}
      %
      \fill[c1] ([xshift=2cm,yshift=-8cm]current page.center)
        rectangle ++ (-12.7,7.7);
      \node[text=white,anchor=west,scale=3,inner sep=0pt] at
      ([xshift=-9.5cm,yshift=-3.25cm]current page.center) {\@title};
      \node[text=white,anchor=west,scale=1.5,inner sep=0pt] at
      ([xshift=-9.5cm,yshift=-6cm]current page.center) {\authorname:\ \@author};
      \node[text=white,anchor=west,scale=1.5,inner sep=0pt] at
      ([xshift=-9.5cm,yshift=-6.75cm]current page.center)%
      {\textbookname:\ \@textbook\ (\@textbookauthor\ \@textbookversion)};
      %
      % \draw[gray,line width=5mm]
      % ([xshift=2mm,yshift=-1mm]current page.south west) 
        rectangle ([xshift=-2mm,yshift=1mm]current
      % page.north east);
    \end{tikzpicture}
    \clearpage\thispagestyle{empty}
    \begin{center}
      \Large\bfseries Declaration
    \end{center}
    The class file \texttt{mathexercise.cls} for this document
    is available at my github repository
    \href{https://github.com/SwitWu/MathExercise}{MathExercise}.
    Please report any issues you find in this document to me
    through email or github issues. Fork and pull requests are welcome.
    \clearpage
  }
\else\fi
%    \end{macrocode}
% \end{macro}
% \Finale
\endinput